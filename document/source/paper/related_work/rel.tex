OCR functions as the major component of the prescription recognition system. The OCR operation usually consists of three main stages: text detection, text recognition, and drug name extraction. In this section, we briefly present each group's state-of-the-art algorithms and evaluate the method's appropriateness for our system.
\subsection{Text Detection and Recognition}

Text detection is the process of detecting and isolating text from the background of an image. Two typically used text detection methods are the traditional and deep learning methods. Traditional methods such as Scale Invariant Feature Transform (SIFT), Hough transforms, Features from Accelerated Segment Test (FAST) often extract text features manually. Therefore, this method is often complicated, quickly accumulates errors, and has many classification rules that must be manually optimized. With the evolution of deep neural networks in recent years, %deep learning models can yield superior results than traditional approaches in processing time and accuracy when trained with a good data set. 
several methods have utilized deep learning techniques, for instance, semantic segmentation and object detection in images. Semantic segmentation extracts blocks of text from a segmentation map created by the FCN and then creates a bounding box through post-processing. Some algorithms that employed this concept include TextSegNet, WordDetNet, and three CNN-based models DNet, SNet, CNet. 
%This approach can handle multi-oriented text, but individual text lines or words near each other need more postfix after the text spotting step. 
Another approach for text recognition in images takes inspiration from general object detection neural models, which are listed and examined in detail in this comprehensive survey \cite{ye2014text}. This method aims at recognizing the text in the scene, with its essence being to associate the texts as objects to predict the bounding box. 
%In addition, the direction will also be indicated to align the axis for the bounding box for multi-oriented texts.
Some generally used methods that leverage this approach are CTPN \cite{tian2016detecting}, CRAFT \cite{baek2019character}, EAST, SegLink.

Once the text is detected, the text recognition phase converts the recognized text into editable digital text. Recent text recognition methods can be classified into segmentation-based and segmentation-free. In a segmentation-based method, the text images are preprocessed and segmented before using a character classification model to recognize characters before aggregating them into lines. In contrast, the segmentation-free method recognizes the whole line of text at once by converting the text image into the output text string using the encoder-decoder framework.
%which consists of 4 stages. The first step is image preprocessing which removes background and image distortion using techniques such as spatial transformer network (STN), Thin-Plate Spline interpolation (TPS). The second step extracts features by mapping the input text image to a vector representing the features of the character, using methods such as HOG, VGGNet, ResNet. Subsequently, the context information contained in the text string is spotted and modelized using models like BiLSTM, deep one-dimensional CNN, CNN with receptive field correction. The last step predicts the output editable texts using Connectionist Temporal Classification Loss (CTC Loss), or attention mechanism. 
Some crucial techniques for text recognition include CRF (Conditional Random Field), CRNN (Regressive Convolutional Network), ASTER(Contextual Text Recognition Network).

\subsection{Post-OCR}
The text generated from the OCR step include all the content in the prescription. Recently widely used Post-OCR methods have been examined in great detail in this survey \cite{nguyen2021survey}. Since drug names are scientific, context-independent, and sentence-order insignificant terms, a dictionary of related terms is indispensable to extract drug names from the OCR text. 

Karthikeyan et al.\cite{karthikeyan2021ocr}  proposed an OCR post-processing method on medical documents employing a pre-trained model in natural language processing and a deep neural network model RoBERTa. The authors first used the Tesseract OCR tool to extract the text. In the post-processing step, biomedical entities were dismissed using Named Entity Recognition (NER) of the ScispaCy model, which belongs to Natural Language Processing fields. After withdrawing specific entities, the model detects error words and non-vocabulary words using spell checkers. RoBERTA is then employed to choose the most similar candidates to the error words. Considerable similarities between this method and the method of Thompson et al. \cite{thompson2015customised} are using spell checkers to identify error words. The most significant difference is Thompson added medical terms to spell checkers, while Karthikeyan takes advantage of the extensive medical terminology. However, the data set used is UK NHS, which is a private datasets containing patient private information. According to the authors, the method achieves 81\% accuracy without further training on the new datasets. The common feature of the models in these papers \cite{thompson2015customised, schulz2017multi, qader2019diagnosis, karthikeyan2021ocr} is that they are all to solve problems in a particular domain, including biomedicine and literature. Among them, Karthikeyan et al. proposed a post-processing approach using models in natural language processing. The Scispacy model used in this work \cite{karthikeyan2021ocr} could identify medical entities using Rxnorm, a database containing the active ingredients of drugs authorized for circulation in the United States. However, there was a distinction between the drug name and its active ingredient in the prescription we collected. While the drug name is the term given by the company that manufactured the drug, the active ingredient is usually the name of the chemical compound. Our proposed method partially solves this problem by creating a database containing drug names and their chemical compounds.

Our previous work \cite{nguyen2021developing} presented a prescription recognition model based on CRAFT and Tesseract OCR tool. We used fuzzy search based on Levenshtein distance to search for drug names in the database collected from Drugbank Vietnam \cite{drugbank}. However, the remaining issues in this system are the slow speed of drug extraction and the small size of the database. In addition, redundant information such as dosage used, name of the disease often causes noise and affects the speed and accuracy of the system. Therefore, in this paper, we continue to propose an improved model. Specifically, we redesigned the system, changing the models in stages and focusing more on the post-OCR step to increase the model's performance.
