\chapter*{Đề cương chi tiết}
\label{proposal}


\section*{Thông tin chung}

\begin{itemize}
    \item Tên đề tài: Kết hợp so khớp mẫu và học sâu trong bài toán trích xuất thông tin đơn thuốc.
    \item Tên đề tài bằng tiếng Anh: Combination of template matching and deep learning in extracting prescription information.
    \item Giảng viên hướng dẫn: \tenGVHD
    \item Nhóm sinh viên thực hiện:
    \begin{itemize}
        \item Nguyễn Hoàng Đức - MSSV: 18120018
        \item Hà Văn Duy - MSSV: 18120339
    \end{itemize}

    \item Thời gian thực hiện: Từ 01/2022 đến 07/2022
    \item Loại đề tài: Ứng dụng
\end{itemize}

\section*{Nội dung thực hiện}

\subsection*{Giới thiệu đề tài}

Đã hơn một năm kể từ khi bùng phát đại dịch  Covid-19 và lan rộng ra toàn cầu, trở thành một cuộc khủng hoảng trong lĩnh vực y tế. Đại dịch đã khiến hơn hơn 400 triệu ca nhiễm, 6 triệu ca tử vong và những con số vẫn đang tiếp tục tăng. Tổ chức Y tế Thế giới (WHO) gọi Covid-19 là chu kỳ "hoảng loạn - lãng quên". Sự bành trướng này yêu cầu chúng ta phải thích nghi và sống chung với đại dịch. Tuy nhiên, tình hình dịch bệnh phức tạp
   sinh ra những suy nghĩ tiêu cực cho con người như cảm giác sợ hãi, lạc lối. Điều đó đã khiến cho nhiều người đưa ra những quyết định thiếu sáng suốt khi sử dụng các loại thuốc điều trị bệnh, từ  đó gây ra những hậu quả đáng tiếc, nhẹ thì tiền mất tật mang, nặng thì ảnh hưởng đến tính mạng. Để tránh những trường hợp không may trong quá trình sử dụng thuốc, đòi hỏi cần phải có một hệ thống hỗ trợ người dùng sử dụng đơn thuốc cũng như quản lý thông tin thuốc một cách chính xác và an toàn hơn.
    
    Có thể thấy, đơn thuốc là thứ vô cùng quan trọng đối với những bệnh nhân. Nó chứa đựng nhiều thông tin có giá trị như thông tin bệnh nhân, tên bác sĩ thăm khám, tình trạng bệnh nhân và thông tin về thuốc đi kèm với liều lượng. Dựa vào các tên thuốc được kê đơn mà nhà thuốc mới có thể bán đúng thuốc trị bệnh. Tuy nhiên, việc ghi nhớ thông tin trong đơn thuốc lại không phải là điều đơn giản. Cụ thể đối với tên thuốc, nó thường là các từ khóa chuyên ngành, khó phát âm và không liên hệ nhiều tới cuộc sống.
    
    Với sự bùng nổ của cuộc cách mạng công nghệ 4.0, dữ liệu mở rộng rất nhanh dẫn đến nhu cầu có nhiều công cụ để quản lý thông tin hiệu quả. Việc ghi nhớ và lưu trữ thông tin trong đơn thuốc một cách thủ công hiện nay đang tạo ra một rào cản lớn cho sự phát triển của ngành y tế. Việc thất lạc hồ hơ là điều không hiếm thấy, dẫn đến khó khăn trong việc truy vết và theo dõi hồ sơ bệnh án của bệnh nhân, ảnh hưởng trực tiếp tới sức khỏe của người bệnh cũng như bệnh viện hay nhà thuốc.
    
    
    Với những lý do được trình bày ở trên, nhóm chúng tôi quyết định  thực hiện đề tài này, giúp đưa ra một ứng dụng hỗ trợ con người quản lý đơn thuốc hiệu quả hơn, gián tiếp hỗ trợ bệnh nhân giải quyết những sai sót không đáng có. Một lợi ích mà đề tài nhóm chúng tôi đem lại nữa đó là giúp con người tiếp cận dễ dàng hơn với công nghệ hiện đại, góp phần giúp Việt Nam bắt kịp với sự phát triển trên thế giới.

\subsection*{Mục tiêu đề tài}

Mục tiêu đề tài khóa luận của chúng tôi là:
    \begin{itemize}
     \item Đọc và tìm hiểu những nghiên cứu có liên quan gần đây về những mô hình phát hiện ký tự quang học có thể áp dụng vào đề tài như: Tesseract \cite{smith2007overview}, CTPN \cite{tian2016detecting}, CRAFT \cite{baek2019character},...  
     \item Tìm hiểu và ứng dụng công trình phù hợp hơn trong việc nhận dạng kí tự quang học trong tiếng Việt, cụ thể là VietOCR \cite{VietOCR} .
     \item Tìm ra đặc trưng của đơn thuốc, từ đó thiết kế nên một giải pháp trích xuất thông tin tương ứng dựa trên kỹ thuật mẫu.
     \item Xây dựng, cải tiến một số mô hình xử lý ngôn ngữ tự nhiên phục vụ cho tác vụ hậu xử lý văn bản cũng như trích xuất thuốc.
     \tiem Xây dựng mô hình giải quyết vấn đề nhập nhằng thông tin trong đơn thuốc.
     \item Xây dựng một ứng dụng hỗ trợ người dùng trong quá trình quản lý thông tin thuốc trong đơn thuốc như liều lượng, giá thành, thành phần.
     \item Cải thiện hiệu suất của mô hình xử lý đơn thuốc hiện tại tốt hơn những mô hình trước đây. 
     \item Xây dựng một tập dữ liệu đơn thuốc Việt Nam, phục vụ cho quá trình huấn luyện và kiểm chứng mô hình, góp phần cải tiến mô hình sau này.
     \item Nâng cao khả năng đọc hiểu, nghiên cứu tài liệu và kỹ năng làm việc nhóm.
    \end{itemize}

\subsection*{Phạm vi đề tài}

Phạm vi đề tài khóa luận của chúng tôi được giới hạn trong việc tìm hiểu, cải tiến các mô hình nhận diện ký tự quang học hiện nay như CRAFT \cite{baek2019character}, VietOCR \cite{VietOCR}, CTPN \cite{tian2016detecting}. Ngoài ra, nhóm cũng tìm hiểu, xây dựng một mô hình phân lớp thông tin trong thuốc dựa trên xử lý ngôn ngữ tự nhiên. Cuối cùng, nhóm sử dụng kết hợp phương pháp tìm kiếm mờ để sửa lỗi cho những thông tin thuốc bị sai, đảm bảo chất lượng của kết quả mô hình.

\subsection*{Cách tiếp cận}

 Cách tiếp cận dự kiến trong quá trình thực hiện đề tài:
    \begin{itemize}
        \item Chuẩn bị dữ liệu: Tiến hành thu thập dữ liệu, bao gồm dữ liệu các đơn thuốc và dữ liệu tên thuốc độc lập từ internet nhằm phục vụ cho việc huấn luyện và kiểm nghiệm mô hình.
        \item Xây dựng mô hình và thực nghiệm: Tìm kiếm những tài liệu, bài báo khoa học trên các nền tảng như GoogleScholar, Arxiv, IEEE có liên quan đến lĩnh vực nhận diện ký tự quang học và xử lý ngôn ngữ tự nhiên, từ đó chọn ra phương pháp kỹ thuật mẫu đặc trưng và phương pháp học sâu để giải quyết bài toán đề ra. Sau cùng kiểm nghiệm mô hình trên tập dữ liệu đơn thuốc thu thập được.
        \item So sánh và đánh giá mô hình: So sánh và đánh giá hiệu suất của mô hình với các mô hình nhận diện đơn thuốc hiện có dựa trên nhiều độ đo và tham số như Precision, Recall, H-mean, WER, CER và thời gian thực thi cho một dữ liệu đầu vào.
        \item Tạo ra ứng dụng để đưa vào thực tế: Tìm hiểu và ứng dụng kiến thức về môn nhập môn công nghệ phần mềm để tạo ra một ứng dụng chụp ảnh đơn thuốc và gửi về server để xử lý, sau đó gửi trả kết quả là thông tin đơn thuốc cho người dùng. Người dùng có thể dễ dàng quản lý thông tin thuốc này ngay trên thiết bị di động thông minh.
    \end{itemize}

\subsection*{Kết quả đề tài}

Kết quả dự kiến đề ra cho đề tài khóa luận này:
    \begin{itemize}
        \item Mỗi cá nhân trong nhóm có thể hiểu được tổng quan về nhận dạng ký tự quang học (các mô hình như Tesseract \cite{smith2007overview}, CRAFT \cite{baek2019character}, CTPN \cite{tian2016detecting},...) và một vài kỹ thuật xử lý ngôn ngữ tự nhiên trong việc trích xuất thông tin trong văn bản (pattern extraction, machine learning), có được kỹ năng hiểu và đưa ra quy trình giải quyết vấn đề khi gặp một bài toán liên quan đến những lĩnh vực này.
        \item Đề xuất được một mô hình nhận diện đơn thuốc dựa trên kết quả nghiên cứu, có hiệu suất tốt hơn, cải thiện lên tới 30\% cho kết quả thực nghiệm đối với độ đo H-mean, và giảm một nửa thời gian thực thi so với những mô hình đã có từ trước, giúp mô hình chúng tôi đề xuất có tính thực tế cao và có thể ứng dụng vào cuộc sống.
        \item Đưa lên Google Play một ứng dụng nhận diện đơn thuốc kết hợp lưu trữ thông tin thuốc và được người dùng đón nhận, đánh giá tốt.
    \end{itemize}

\subsection*{Kế hoạch thực hiện}

Kế hoạch thực hiện đề tài khóa luận được trình bày trong bảng sau:

 \begin{table}
    \centering
    \normalsize  
    \begin{adjustbox}{angle=-90}
    \begin{tabular}{|c|c|c|}
    
        \hline 
        Thời gian & Hà Văn Duy & Nguyễn Hoàng Đức \\\hline
        
        1/2022 & \multicolumn{2}{|c|}{Bắt đầu nhận đề tài khoa học và lập nhóm.} \\\hline
        
        1/2022 - 2/2022 & \multicolumn{2}{|c|}{Tìm hiểu các công trình nghiên cứu liên quan. Báo cáo với giảng viên hàng tuần.} \\\hline
        
        2/2022 - 3/2022 & Đề xuất mô hình Xử lý ngôn ngữ & Đề xuất mô hình Nhận dạng ký tự quang học \\\hline
        3/2022 - 4/2022 & \multicolumn{2}{|c|}{Cài đặt mô hình và thử nghiệm trên các bộ dữ liệu đã thu thập.} \\\hline
        
        4/2022 - 5/2022 &  \multicolumn{2}{|c|}{Tối ưu mô hình và viết paper nộp cho một số hội nghị quốc tế.} \\\hline
        5/2022 - 6/2022 & \multicolumn{2}{|c|}{Lập trình ứng dụng nhận diện đơn thuốc và đưa lên Google Play.} \\\hline
        6/2022 - 7/2022 & \multicolumn{2}{|c|}{Viết báo cáo khóa luận và báo cáo kết quả.} \\\hline
    \end{tabular}
    \end{adjustbox}
    \caption{Kế hoạch thực hiện khóa luận}
    
    \end{table}
    
    
