\chapter*{Tóm tắt}
\label{abstract}

Nhận dạng tên thuốc từ đơn thuốc đang là bài toán quan trọng, được nhiều người quan tâm trong bối cảnh sức khỏe của con người đang trở thành vấn đề chính mà xã hội có nhu cầu theo dõi, chăm sóc và phát triển, nhất là sau đại dịch. Việc trích xuất và lưu trữ lại tên thuốc trong đơn thuốc mở ra nhiều cơ hội cho không chỉ bản thân người bệnh, mà còn đóng vai trò quan trọng trong sự phát triển và tự động hóa của đời sống xã hội.

Dựa trên nghiên cứu trước đó, khóa luận này tập trung tiến hành nghiên cứu, thử nghiệm và triển khai mô hình mang tên \textit{Nhận dạng đơn thuốc từ đơn thuốc} (\codeword{MEP}) nhằm nhận dạng có hiệu quả tên thuốc từ các đơn thuốc được chụp bởi các thiết bị di động, máy ảnh. Mô hình đề xuất của chúng tôi khai thác thêm một số luật heuristic dựa trên mẫu đặc trưng và mô hình mạng tích chập giãn nở nhằm trích xuất các thông tin cần thiết từ văn bản được trả ra từ kết quả OCR. Mô hình chúng tôi cho ra kết quả rất ấn tượng khi so sánh với phiên bản tiền nghiệm, đạt 0.82 điểm \textbf{H-mean} trên tập dữ liệu đơn thuốc mà chúng tôi thu thập, đề xuất. Ngoài ra, \codeword{MEP} cũng giúp tăng cường hiệu suất nhận dạng và trích xuất tên thuốc, với thời gian xử lý trung bình giảm 2.6 lần, chỉ từ 6.67 giây cho mỗi đơn thuốc.
