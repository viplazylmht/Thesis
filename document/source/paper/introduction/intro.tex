Prescriptions are common for individuals and organizations involved in the medical field, especially patients. It contains a lot of critical information, including the patient's identity, the doctor's name, the name of the disease, the current condition and most importantly, the name of the medicine and the dosage used. Despite the details, the prescription does not clearly explain the origin, ingredients, and side effects that the drug brings to the user. This has led to an increase in the number of patients poisoning or, most seriously, death (due to side effects or allergies to some drug ingredients). In certain circumstances, improper medication usage might exacerbate chronic conditions. Furthermore, certain medicines may have major side effects on the body's organs. Depending on the drug administered, the body might heal or suffer long-term physical damage. The heart, liver, and kidneys are organ systems that are extremely vulnerable to harm.

According to statistics, from 1968 to 2019 in the US, 1,015,060 people died from a drug overdose. In addition, according to federal data released on November 17, the US recorded over 100,000 drug overdose deaths for the first time in the past 12 months, an increase of 28.5\% over the same period before \footnote{\url{https://edition.cnn.com/2021/11/17/health/drug-overdose-deaths-record-high/index.html}}.

% In addition, users may want to thoroughly lookup drug names on popular search engines such as Google, Bing. However, because prescriptions and drug names often contain quite specialized words, it inevitably causes specific difficulties and complications for users in the search process. There is a desire for a simple and easy-to-use system for looking up and identifying prescriptions, as well as a tool for managing drugs and the user's drug usage process.

For these mentioned reasons, a system for Medicines Extraction on Prescriptions (MEP) is introduced. Our user-friendly system has a low computational cost and fast processing time, making it widely-accepted by patients and medical facilities worldwide. Generally, MEP is a hybrid of recent popular approaches including OCR, Post-OCR, pattern matching, and natural language processing. This paper is organized as follows: problem statement is given in Section 1. Section 2 demonstrates the architecture of related techniques. Background technique and proposed recognition system are described in Sections 3 and Section 4, respectively. Experimental results on our datasets are presented in Section 5, and Section 6 provides conclusions and future work.

